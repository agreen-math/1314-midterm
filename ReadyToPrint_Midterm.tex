\documentclass[addpoints]{exam}

\usepackage[utf8]{inputenc}
\usepackage{array}
\usepackage{graphicx}
\usepackage{multicol}
\usepackage{amsmath}
\usepackage{paracol}
\usepackage{pgf,tikz}
\usepackage{mathrsfs}
\usetikzlibrary{arrows}

\newcommand{\ci}{\textbf{Chapter 1}}
\newcommand{\cii}{\textbf{Chapter 2}}
\newcommand{\ciii}{\textbf{Chapter 3}}
\newcommand{\civ}{\textbf{Chapter 4}}
\newcommand{\cv}{\textbf{Chapters 5 and 6}}

\setlength\answerskip{2ex}
\setlength\answerlinelength{3in}

% Toggle solutions
%\printsolutions
\noprintsolutions

\pagestyle{headandfoot}
\runningheadrule
\firstpageheader{Math 1314}{Midterm Exam (WEB)}{Fall 2025}
\runningheader{Math 1314}{Midterm Exam (WEB)}{Fall 2025}
\runningheadrule
\firstpagefooter{}{}{}
\runningfooter{}{}{}

\begin{document}

\noindent{Number of Questions: \numquestions\hspace{\stretch{1}} Point Total: \numpoints}
\begin{questions}
\question Evaluate the function \(f(x) = -x^{2} + 5 \, x + 9\).\\\begin{solution}\ci\end{solution}

\begin{parts}
  \part[3] Find \(f(10)\).

\begin{solution}
\(-1(10)^2 + 5(10) + 9\)

\(\boxed{f(10) = -41}\)
\end{solution}  \vspace{\stretch{1}} \\ \answerline
  \part[3] Find \(f(-x)\).

\begin{solution}
\(-1(-x)^2 + 5(-x) + 9\)

\(\boxed{f(-x) = -x^{2} - 5 \, x + 9}\)
\end{solution}  \vspace{\stretch{1}} \\ \answerline
  \part[4] Find \(f(x + a)\).

\begin{solution}
\(-1(x+a)^2 + 5(x+a) + 9\)

\(\boxed{f(x + a) = -a^{2} - 2 \, a x - x^{2} + 5 \, a + 5 \, x + 9}\)
\end{solution}  \vspace{\stretch{1}} \\ \answerline
\end{parts}

\question[10] \textbf{\textit{Without solving}}, use the discriminant to determine the number and the type of the solutions.\\\begin{solution}\ciii\end{solution} \vspace{\stretch{3}}\\
number of solutions: \fillin \hspace{\stretch{1}} type of solutions: \fillin[][2.5in] \newpage
% Original: Determine the number and type of solutions for the following equation: 

 \(x^{2} + 9 \, x - 3 = -2 \, x\) 

\begin{solution}
\(1x^2 + (11)x + -3 = 0\) 

 \(\Delta = (11)^2 - 4(1)(-3)\) 

 \(\Delta = 133\) 

 \(\boxed{\text{ two solutions, unequal real roots }}\)
\end{solution}

\uplevel{For the following questions, solve for \textbf{all} solutions. Identify any extraneous solutions.}
\question[10] Solve: Solve for all solutions. Identify any extraneous solutions.

 \(x^{2} + 8 \, x + 19 = 0\) 

\begin{solution}
\(x = \frac{-(8) \pm \sqrt{(8)^2 - 4(1)(19)}}{2(1)}\) 

 \(= \frac{-8 \pm \sqrt{64 - 76}}{2}\) 

 \(= \frac{-8 \pm \sqrt{-12}}{2}\) 

 \(= \frac{-8 \pm 2i\sqrt{3}}{2}\) 

 \(= \frac{2(-4 \pm i\sqrt{3})}{2}\) 

 \(\boxed{x = -4 \pm i\sqrt{3}}\) 

 (No extraneous solutions)
\end{solution}\\\begin{solution}\cii\end{solution} \vspace{\stretch{1}}\\\,\answerline

\question[10] Solve: Solve for all solutions. Identify any extraneous solutions.

 \(\sqrt{x + 16} + 2 = 5\) 

\begin{solution}
\(\sqrt{x + 16} = 3\) 

 \(x + 16 = 9\) 

 \(x = 9 - 16\) 

 \(\boxed{x = -7}\) 

 (No extraneous solutions)
\end{solution}\\\begin{solution}\cii\end{solution} \vspace{\stretch{1}}\\\,\answerline

\question[10] Solve: Solve for all solutions. Identify any extraneous solutions. 

 \(\displaystyle \frac{x}{x + 1} - \frac{3}{(x + 1)(x - 2)} = \frac{-7}{x - 2}\) 

\begin{solution}
\(x(x - 2) - 3 = -7(x + 1)\) 

 \(x^{2} + 5 \, x + 4 = 0\) 

 \((x + 1)(x + 4) = 0\) 

 \(x = -4, \quad x = -1\) 

 \(x = -1 \implies \text{Extraneous}\) 

 \(\boxed{x = -4}\)
\end{solution}\\\begin{solution}\cii\end{solution} \vspace{\stretch{1}}\\\,\answerline \newpage

\question[10] Given that the function is one-to-one, find the inverse function \(f^{-1}(x)\). 

 \(f(x) = \sqrt[3]{x + 1} + 5\) 

\begin{solution}
\(y = f(x) = \sqrt[3]{x + 1} + 5\) 

 \(x = \sqrt[3]{y + 1} + 5\) 

 \(x - 5 = \sqrt[3]{y + 1}\) 

 \((x - 5)^3 = y + 1\) 

 \(y = (x - 5)^3 - 1\) 

 \(\boxed{ f^{-1}(x) = (x - 5)^3 - 1 }\) 

  

 \(= (x - 5)(x^2 - 10x + 25) - 1\) 

 \(\boxed{ f^{-1}(x) = x^{3} - 15 \, x^{2} + 75 \, x - 126 }\)
\end{solution}\\\begin{solution}\ci\end{solution} \vspace{\stretch{2}}\\\,\answerline \newpage

\question Suppose a rocket carrying fireworks is launched from a hill 50 feet above a lake. The rocket's height \(h\) (in feet) above the lake at time \(t\) (in seconds) is given by

 \(h(t) = -16t^2 + 128t + 50\)
\begin{parts}
  \part[2] When will the rocket reach its maximum height?

\begin{solution}
\(t = \frac{-128}{2(-16)} = \frac{-128}{-32}\) 

 \(\boxed{t = 4 \text{ sec}}\)
\end{solution}  \fillwithlines{0.5in}
  \part[2] What is the maximum height the rocket will reach?

\begin{solution}
\(h(4) = -16(4)^2 + 128(4) + 50\) 

 \(= -256 + 512 + 50\) 

 \(\boxed{h = 306 \text{ ft}}\)
\end{solution}  \fillwithlines{0.5in}
  \part[1] Interpret your results for the previous two questions in one or more complete sentences, including appropriate units.

\begin{solution}
The rocket will reach a maximum height of 306 ft 4 seconds after launch.
\end{solution}  \fillwithlines{0.5in}
\end{parts}

\question Find the following compositions:Given \(f(x) = -3 \, x\) and \(g(x) = -2 \, x^{2} + 6\), find the following:\\\begin{solution}\ci\end{solution}

\begin{parts}
  \part[6] \((f \circ g)(x)\)

\begin{solution}
\(f(g(x)) = -3(-2 \, x^{2} + 6)\) 

 \(\boxed{ 6 \, x^{2} - 18 }\) 

ERROR 1: \(g(f(x))\)

 \(= -2(-3 \, x)^2 + 6\) 

 \(= -2(9 \, x^{2}) + 6\) 

 \(\boxed{ -18 \, x^{2} + 6 }\) 

ERROR 2: \(f(x) \cdot g(x)\)

 \(= (-3 \, x)(-2 \, x^{2} + 6)\) 

 \(\boxed{ 6 \, x^{3} - 18 \, x }\)
\end{solution}  \vspace{\stretch{1}} \\ \answerline
  \part[4] \((f \circ f)(x)\)

\begin{solution}
\(f(f(x)) = -3(-3 \, x)\) 

 \(\boxed{ 9 \, x }\)
\end{solution}  \vspace{\stretch{1}} \\ \answerline
\end{parts} \newpage

\question[5] Evaluate the difference quotient.\\\begin{solution}\ci\end{solution}
Evaluate the difference quotient, \(\displaystyle{ \frac{f(x+h)-f(x)}{h}}\) , for \(f(x) = 4 \, x^{2} - 8\). 

\begin{solution}
\(f(x+h) = 4(x+h)^2 - 8\) 

 \(= 4(x^2 + 2xh + h^2) - 8\) 

 \(= 4 \, h^{2} + 8 \, h x + 4 \, x^{2} - 8\) 

 \(\displaystyle{\frac{f(x+h)-f(x)}{h} = \frac{ (4 \, h^{2} + 8 \, h x + 4 \, x^{2} - 8) - (4 \, x^{2} - 8) }{h}}\) 

 \(=\displaystyle{ \frac{ 4 \, h^{2} + 8 \, h x }{h}}\) 

 \(= \displaystyle{ \frac{ h(4 \, h + 8 \, x) }{h}}\) 

 \(\boxed{ 4 \, h + 8 \, x }\)
\end{solution} \vspace{\stretch{1}} \\ \answerline \newpage

\question[5] Find the vertex and properties.
Find each of the properties below for the given function: 

 \(f(x) = \frac{1}{2}x^2 -8\) 

\begin{itemize}
\item
Vertex

\item
Axis of Symmetry

\item
\(x\)-intercepts

\item
\(y\)-intercept

\item
Domain

\item
Range

\end{itemize}
\begin{solution}
\(\textbf{Direction: } \text{Opens Up}\) 

 \(\textbf{Vertex: } (0, -8)\) 

 \(\textbf{Axis of Symmetry: } x = 0\) 

 \(\textbf{x-intercepts:}\) 

 \(\frac{1}{2}x^2 -8 = 0\) 

 \(x^2 = 16\) 

 \(x = 0 \pm 4\) 

 \(\boxed{ -4, 4 }\) 

 \(\textbf{y-intercept:}\) 

 \(= \frac{1}{2}(0)^2 -8\) 

 \(= \frac{1}{2}(0) -8\) 

 \(= 0 -8\) 

 \(\boxed{ -8 }\) 

 \(\textbf{Domain: } (-\infty, \infty)\) 

 \(\textbf{Range: } [-8, \infty)\)
\end{solution}

\question[5] Use the characteristics to graph.
A quadratic function has the characteristics given below. Use the axis of symmetry to generate two additional points, then use all five points to graph the function. 

\begin{itemize}
\item
\(\text{Vertex: } (0, 1)\)

\item
\(x\text{-intercept: } -1\)

\item
\(y\text{-intercept: } 1\)

\end{itemize}
 \begin{tikzpicture}[scale=0.35] \draw[step=1cm, gray!40, very thin] (-10,-10) grid (10,10); \draw[thick, <->] (-10.5,0) -- (10.5,0); \draw[thick, <->] (0,-10.5) -- (0,10.5); \end{tikzpicture} 

\begin{solution}
\begin{multicols}{2}  

\begin{itemize}
\item
\(\text{Vertex: } (0, 1)\)

\item
\(y\text{-intercept: } (0, 1)\)

\item
\(\text{Reflected } y\text{-int: } (0, 1)\)

\item
\(\text{Given Root: } (-1, 0)\)

\item
\(\text{Reflected Root: } (1, 0)\)

\end{itemize}
 \columnbreak  

 \begin{tikzpicture}[scale=0.35] \draw[step=1cm, gray!40, very thin] (-10,-10) grid (10,10); \draw[thick, <->] (-10.5,0) -- (10.5,0); \draw[thick, <->] (0,-10.5) -- (0,10.5); \clip (-10,-10) rectangle (10,10); \draw[line width=1.5pt, blue!80!black, samples=100, domain=-10:10, <->] plot (\x, {-1.0*(\x - 0)^2 + 1}); \fill[blue] (0,1) circle (7pt); \fill[blue] (-1,0) circle (7pt); \fill[blue] (1,0) circle (7pt); \fill[blue] (0,1) circle (7pt); \fill[blue] (0,1) circle (7pt); \end{tikzpicture} 

 \end{multicols}
\end{solution} \newpage

\question[10] Identify transformations and graph.
Use the table to identify the transformations described by \(g(x) = f(x + 2) + 5\). 

 Circle the option that applies and fill in the blanks as appropriate. Then apply these transformations to the graph of the function shown below. 

 \(\renewcommand{\arraystretch}{3} \begin{array}{|l|l|} \hline \textbf{Horizontal Transformations} & \textbf{Vertical Transformations} \\ \hline \text{Reflection: YES or NO} & \text{Reflection: YES or NO} \\ \hline \text{Dilation: \underline{\hspace{2cm}} times as wide} & \text{Dilation: \underline{\hspace{2cm}} times as tall} \\ \hline \text{Translation: \underline{\hspace{2cm}} units LEFT or RIGHT} & \text{Translation: \underline{\hspace{2cm}} units UP or DOWN} \\ \hline \end{array}\) 

 \noindent\makebox[\textwidth][c]{ \begin{minipage}{0.48\textwidth} \centering \begin{tikzpicture}[scale=0.39] \draw[step=1cm, gray!40, very thin] (-10,-10) grid (10,10); \draw[thick, <->] (-10.5,0) -- (10.5,0); \draw[thick, <->] (0,-10.5) -- (0,10.5); \draw[line width=1.5pt, black] (-2,1) -- (-1,2) -- (0,0) -- (1,4) -- (3,4);\fill[black] (-2,1) circle (5pt); \fill[black] (-1,2) circle (5pt); \fill[black] (0,0) circle (5pt); \fill[black] (1,4) circle (5pt); \fill[black] (3,4) circle (5pt); \end{tikzpicture} \end{minipage} \hfill \begin{minipage}{0.48\textwidth} \centering \begin{tikzpicture}[scale=0.39] \draw[step=1cm, gray!40, very thin] (-10,-10) grid (10,10); \draw[thick, <->] (-10.5,0) -- (10.5,0); \draw[thick, <->] (0,-10.5) -- (0,10.5); \end{tikzpicture} \end{minipage} } 

\begin{solution}
\begin{itemize}
\item
 \(\textbf{Horizontal: } \text{Reflection: NO, Dilation: 1, Shift: 2 units LEFT.}\) 

\item
 \(\textbf{Vertical: } \text{Reflection: NO, Dilation: 1, Shift: 5 units UP.}\) 

\end{itemize}


 \begin{tikzpicture}[scale=0.39] \draw[step=1cm, gray!40, very thin] (-10,-10) grid (10,10); \draw[thick, <->] (-10.5,0) -- (10.5,0); \draw[thick, <->] (0,-10.5) -- (0,10.5); \draw[line width=1.5pt, blue] (-4,6) -- (-3,7) -- (-2,5) -- (-1,9) -- (1,9);\fill[blue] (-4,6) circle (5pt); \fill[blue] (-3,7) circle (5pt); \fill[blue] (-2,5) circle (5pt); \fill[blue] (-1,9) circle (5pt); \fill[blue] (1,9) circle (5pt); \end{tikzpicture}
\end{solution} \newpage


\end{questions}
\end{document}
