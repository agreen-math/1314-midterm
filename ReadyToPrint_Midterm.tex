\documentclass[addpoints]{exam}

\usepackage[utf8]{inputenc}
\usepackage{array}
\usepackage{graphicx}
\usepackage{multicol}
\usepackage{amsmath}
\usepackage{paracol}
\usepackage{pgf,tikz}
\usepackage{mathrsfs}
\usetikzlibrary{arrows}

\newcommand{\ci}{\textbf{Chapter 1}}
\newcommand{\cii}{\textbf{Chapter 2}}
\newcommand{\ciii}{\textbf{Chapter 3}}
\newcommand{\civ}{\textbf{Chapter 4}}
\newcommand{\cv}{\textbf{Chapters 5 and 6}}

\setlength\answerskip{2ex}
\setlength\answerlinelength{3in}

% Toggle solutions
%\printsolutions
\noprintsolutions

\pagestyle{headandfoot}
\runningheadrule
\firstpageheader{Math 1314}{Midterm Exam (WEB)}{Fall 2025}
\runningheader{Math 1314}{Midterm Exam (WEB)}{Fall 2025}
\runningheadrule
\firstpagefooter{}{}{}
\runningfooter{}{}{}

\begin{document}

\noindent{Number of Questions: \numquestions\hspace{\stretch{1}} Point Total: \numpoints}
\begin{questions}
\question Evaluate the function \(f(x) = 2 \, x^{2} + 4 \, x + 9\).
\begin{parts}
  \part[3] Find \(f(7)\).

\begin{solution}
\(2(7)^2 + 4(7) + 9\)

\(\boxed{f(7) = 135}\)
\end{solution}  \vspace{\stretch{1}} \\ \answerline
  \part[3] Find \(f(-x)\).

\begin{solution}
\(2(-x)^2 + 4(-x) + 9\)

\(\boxed{f(-x) = 2 \, x^{2} - 4 \, x + 9}\)
\end{solution}  \vspace{\stretch{1}} \\ \answerline
  \part[4] Find \(f(x + a)\).

\begin{solution}
\(2(x+a)^2 + 4(x+a) + 9\)

\(\boxed{f(x + a) = 2 \, a^{2} + 4 \, a x + 2 \, x^{2} + 4 \, a + 4 \, x + 9}\)
\end{solution}  \vspace{\stretch{1}} \\ \answerline
\end{parts}

\question[10] \textbf{\textit{Without solving}}, use the discriminant to determine the number and the type of the solutions.
Determine the number and type of solutions for the following equation: 

 \(-2 \, x^{2} - x + 2 = -6 \, x\) 

\begin{solution}
\(-2x^2 + (5)x + 2 = 0\) 

 \(\Delta = (5)^2 - 4(-2)(2)\) 

 \(\Delta = 41\) 

 \(\boxed{\text{ two solutions, unequal real roots }}\)
\end{solution} \vspace{\stretch{3}}\\
number of solutions: \fillin \hspace{\stretch{1}} type of solutions: \fillin[][2.5in] \newpage

\uplevel{For the following questions, solve for \textbf{all} solutions. Identify any extraneous solutions.}
\question[10] Solve: Solve for all solutions. Identify any extraneous solutions.

 \(x^{2} + 8 \, x + 19 = 0\) 

\begin{solution}
\(x = \frac{-(8) \pm \sqrt{(8)^2 - 4(1)(19)}}{2(1)}\) 

 \(= \frac{-8 \pm \sqrt{64 - 76}}{2}\) 

 \(= \frac{-8 \pm \sqrt{-12}}{2}\) 

 \(= \frac{-8 \pm 2i\sqrt{3}}{2}\) 

 \(= \frac{2(-4 \pm i\sqrt{3})}{2}\) 

 \(\boxed{x = -4 \pm i\sqrt{3}}\) 

 (No extraneous solutions)
\end{solution} \vspace{\stretch{1}}\\\,\answerline

\question[10] Solve: Solve for all solutions. Identify any extraneous solutions.

 \(\sqrt{x + 17} - 5 = 0\) 

\begin{solution}
\(\sqrt{x + 17} = 5\) 

 \(x + 17 = 25\) 

 \(x = 25 - 17\) 

 \(\boxed{x = 8}\) 

 (No extraneous solutions)
\end{solution} \vspace{\stretch{1}}\\\,\answerline

\question[10] Solve: Solve for all solutions. Identify any extraneous solutions. 

 \(\displaystyle \frac{x}{x + 2} + \frac{6}{(x + 2)(x + 5)} = \frac{9}{x + 5}\) 

\begin{solution}
\(x(x + 5) + 6 = 9(x + 2)\) 

 \(x^{2} - 4 \, x - 12 = 0\) 

 \((x + 2)(x - 6) = 0\) 

 \(x = 6, \quad x = -2\) 

 \(x = -2 \implies \text{Extraneous}\) 

 \(\boxed{x = 6}\)
\end{solution} \vspace{\stretch{1}}\\\,\answerline \newpage

\question[10] Given that the function is one-to-one, find the inverse function \(f^{-1}(x)\). 

 \(f(x) = \sqrt[3]{x + 2} - 1\) 

\begin{solution}
\(y = f(x) = \sqrt[3]{x + 2} - 1\) 

 \(x = \sqrt[3]{y + 2} - 1\) 

 \(x + 1 = \sqrt[3]{y + 2}\) 

 \((x + 1)^3 = y + 2\) 

 \(y = (x + 1)^3 - 2\) 

 \(\boxed{ f^{-1}(x) = (x + 1)^3 - 2 }\) 

  

 \(= (x + 1)(x^2 + 2x + 1) - 2\) 

 \(\boxed{ f^{-1}(x) = x^{3} + 3 \, x^{2} + 3 \, x - 1 }\)
\end{solution} \vspace{\stretch{2}}\\\,\answerline \newpage

\newpage \question Suppose a rocket carrying fireworks is launched from a hill 53 feet above a lake. The rocket's height \(h\) (in feet) above the lake at time \(t\) (in seconds) is given by

 \(h(t) = -16t^2 + 32t + 53\)
\begin{parts}
  \part[2] When will the rocket reach its maximum height?

\begin{solution}
\(t = \frac{-32}{2(-16)} = \frac{-32}{-32}\) 

 \(\boxed{t = 1 \text{ sec}}\)
\end{solution}  \vspace{\stretch{1}} \\ \answerline
  \part[2] What is the maximum height the rocket will reach?

\begin{solution}
\(h(1) = -16(1)^2 + 32(1) + 53\) 

 \(= -16 + 32 + 53\) 

 \(\boxed{h = 69 \text{ ft}}\)
\end{solution}  \vspace{\stretch{1}} \\ \answerline
  \part[1] Interpret your results for the previous two questions in one or more complete sentences, including appropriate units.

\begin{solution}
The rocket will reach a maximum height of 69 ft 1 seconds after launch.
\end{solution}  \fillwithlines{1in}
\end{parts} \newpage

\question Suppose a rocket carrying fireworks is launched from a hill 63 feet above a lake. The rocket's height \(h\) (in feet) above the lake at time \(t\) (in seconds) is given by

 \(h(t) = -16t^2 + 160t + 63\)
\begin{parts}
  \part[6] When will the rocket reach its maximum height?

\begin{solution}
\(t = \frac{-160}{2(-16)} = \frac{-160}{-32}\) 

 \(\boxed{t = 5 \text{ sec}}\)
\end{solution}  \vspace{\stretch{1}} \\ \answerline
  \part[4] What is the maximum height the rocket will reach?

\begin{solution}
\(h(5) = -16(5)^2 + 160(5) + 63\) 

 \(= -400 + 800 + 63\) 

 \(\boxed{h = 463 \text{ ft}}\)
\end{solution}  \vspace{\stretch{1}} \\ \answerline
  \part[1] Interpret your results for the previous two questions in one or more complete sentences, including appropriate units.

\begin{solution}
The rocket will reach a maximum height of 463 ft 5 seconds after launch.
\end{solution}  \vspace{\stretch{1}} \\ \answerline
\end{parts} \newpage

Given \(f(x) = 3 \, x\) and \(g(x) = -2 \, x^{2} - 1\), find the following:
\begin{parts}
  \part[1] \((f \circ g)(x)\)

\begin{solution}
\(f(g(x)) = 3(-2 \, x^{2} - 1)\) 

 \(\boxed{ -6 \, x^{2} - 3 }\) 

ERROR 1: \(g(f(x))\)

 \(= -2(3 \, x)^2 - 1\) 

 \(= -2(9 \, x^{2}) - 1\) 

 \(\boxed{ -18 \, x^{2} - 1 }\) 

ERROR 2: \(f(x) \cdot g(x)\)

 \(= (3 \, x)(-2 \, x^{2} - 1)\) 

 \(\boxed{ -6 \, x^{3} - 3 \, x }\)
\end{solution} 
  \part[1] \((f \circ f)(x)\)

\begin{solution}
\(f(f(x)) = 3(3 \, x)\) 

 \(\boxed{ 9 \, x }\)
\end{solution} 
\end{parts} \vspace{\stretch{1}} \\ \answerline \newpage

\question[5] Evaluate the difference quotient, \(\displaystyle{ \frac{f(x+h)-f(x)}{h}}\) , for \(f(x) = 5 \, x^{2} - 1\). 

\begin{solution}
\(f(x+h) = 5(x+h)^2 - 1\) 

 \(= 5(x^2 + 2xh + h^2) - 1\) 

 \(= 5 \, h^{2} + 10 \, h x + 5 \, x^{2} - 1\) 

 \(\displaystyle{\frac{f(x+h)-f(x)}{h} = \frac{ (5 \, h^{2} + 10 \, h x + 5 \, x^{2} - 1) - (5 \, x^{2} - 1) }{h}}\) 

 \(=\displaystyle{ \frac{ 5 \, h^{2} + 10 \, h x }{h}}\) 

 \(= \displaystyle{ \frac{ h(5 \, h + 10 \, x) }{h}}\) 

 \(\boxed{ 5 \, h + 10 \, x }\)
\end{solution}

\question[5] Find each of the properties below for the given function: 

 \(f(x) = -\frac{1}{4}x^2 +1\) 

\begin{itemize}
\item
Vertex

\item
Axis of Symmetry

\item
\(x\)-intercepts

\item
\(y\)-intercept

\item
Domain

\item
Range

\end{itemize}
\begin{solution}
\(\textbf{Direction: } \text{Opens Down}\) 

 \(\textbf{Vertex: } (0, 1)\) 

 \(\textbf{Axis of Symmetry: } x = 0\) 

 \(\textbf{x-intercepts:}\) 

 \(-\frac{1}{4}x^2 +1 = 0\) 

 \(x^2 = 4\) 

 \(x = 0 \pm 2\) 

 \(\boxed{ -2, 2 }\) 

 \(\textbf{y-intercept:}\) 

 \(= -\frac{1}{4}(0)^2 +1\) 

 \(= -\frac{1}{4}(0) +1\) 

 \(= 0 +1\) 

 \(\boxed{ 1 }\) 

 \(\textbf{Domain: } (-\infty, \infty)\) 

 \(\textbf{Range: } (-\infty, 1]\)
\end{solution} \newpage

\question[10] A quadratic function has the characteristics given below. Use the axis of symmetry to generate two additional points, then use all five points to graph the function. 

\begin{itemize}
\item
\(\text{Vertex: } (1, 9)\)

\item
\(x\text{-intercept: } -2\)

\item
\(y\text{-intercept: } 8\)

\end{itemize}
 \begin{tikzpicture}[scale=0.35] \draw[step=1cm, gray!40, very thin] (-10,-10) grid (10,10); \draw[thick, <->] (-10.5,0) -- (10.5,0); \draw[thick, <->] (0,-10.5) -- (0,10.5); \end{tikzpicture} 

\begin{solution}
\begin{multicols}{2}  

\begin{itemize}
\item
\(\text{Vertex: } (1, 9)\)

\item
\(y\text{-intercept: } (0, 8)\)

\item
\(\text{Reflected } y\text{-int: } (2, 8)\)

\item
\(\text{Given Root: } (-2, 0)\)

\item
\(\text{Reflected Root: } (4, 0)\)

\end{itemize}
 \columnbreak  

 \begin{tikzpicture}[scale=0.35] \draw[step=1cm, gray!40, very thin] (-10,-10) grid (10,10); \draw[thick, <->] (-10.5,0) -- (10.5,0); \draw[thick, <->] (0,-10.5) -- (0,10.5); \clip (-10,-10) rectangle (10,10); \draw[line width=1.5pt, blue!80!black, samples=100, domain=-10:10, <->] plot (\x, {-1.0*(\x - 1)^2 + 9}); \fill[blue] (1,9) circle (7pt); \fill[blue] (-2,0) circle (7pt); \fill[blue] (4,0) circle (7pt); \fill[blue] (0,8) circle (7pt); \fill[blue] (2,8) circle (7pt); \end{tikzpicture} 

 \end{multicols}
\end{solution} \newpage


\end{questions}
\end{document}
