\documentclass[addpoints]{exam}

\usepackage[utf8]{inputenc}
\usepackage{array}
\usepackage{graphicx}
\usepackage{multicol}
\usepackage{amsmath}
\usepackage{paracol}
\usepackage{pgf,tikz}
\usepackage{mathrsfs}
\usetikzlibrary{arrows}

\newcommand{\ci}{\textbf{Chapter 1}}
\newcommand{\cii}{\textbf{Chapter 2}}
\newcommand{\ciii}{\textbf{Chapter 3}}
\newcommand{\civ}{\textbf{Chapter 4}}
\newcommand{\cv}{\textbf{Chapters 5 and 6}}

\setlength\answerskip{2ex}
\setlength\answerlinelength{3in}

% Toggle solutions
%\printsolutions
\noprintsolutions

\pagestyle{headandfoot}
\runningheadrule
\firstpageheader{Math 1314}{Midterm Exam (WEB)}{Fall 2025}
\runningheader{Math 1314}{Midterm Exam (WEB)}{Fall 2025}
\runningheadrule
\firstpagefooter{}{}{}
\runningfooter{}{}{}

\begin{document}

\noindent{Number of Questions: \numquestions\hspace{\stretch{1}} Point Total: \numpoints}
\begin{questions}
\question Evaluate the function \(f(x) = -4 \, x^{2} - 3 \, x + 1\).\\\begin{solution}\ci\end{solution}

\begin{parts}
  \part[3] Find \(f(7)\).

\begin{solution}
\(-4(7)^2 + -3(7) + 1\)

\(\boxed{f(7) = -216}\)
\end{solution}  \vspace{\stretch{1}} \\ \answerline
  \part[3] Find \(f(-x)\).

\begin{solution}
\(-4(-x)^2 + -3(-x) + 1\)

\(\boxed{f(-x) = -4 \, x^{2} + 3 \, x + 1}\)
\end{solution}  \vspace{\stretch{1}} \\ \answerline
  \part[4] Find \(f(x + a)\).

\begin{solution}
\(-4(x+a)^2 + -3(x+a) + 1\)

\(\boxed{f(x + a) = -4 \, a^{2} - 8 \, a x - 4 \, x^{2} - 3 \, a - 3 \, x + 1}\)
\end{solution}  \vspace{\stretch{1}} \\ \answerline
\end{parts}

\question[10] \textbf{\textit{Without solving}}, use the discriminant to determine the number and the type of the solutions.\\\begin{solution}\ciii\end{solution} \vspace{\stretch{3}}\\
number of solutions: \fillin \hspace{\stretch{1}} type of solutions: \fillin[][2.5in] \newpage
% Original: Determine the number and type of solutions for the following equation: 

 \(4 \, x^{2} + x - 9 = 6 \, x\) 

\begin{solution}
\(4x^2 + (-5)x + -9 = 0\) 

 \(\Delta = (-5)^2 - 4(4)(-9)\) 

 \(\Delta = 169\) 

 \(\boxed{\text{ two solutions, unequal real roots }}\)
\end{solution}

\uplevel{For the following questions, solve for \textbf{all} solutions. Identify any extraneous solutions.}
\question[10] Solve: Solve for all solutions. Identify any extraneous solutions.

 \(x^{2} + 10 \, x + 28 = 0\) 

\begin{solution}
\(x = \frac{-(10) \pm \sqrt{(10)^2 - 4(1)(28)}}{2(1)}\) 

 \(= \frac{-10 \pm \sqrt{100 - 112}}{2}\) 

 \(= \frac{-10 \pm \sqrt{-12}}{2}\) 

 \(= \frac{-10 \pm 2i\sqrt{3}}{2}\) 

 \(= \frac{2(-5 \pm i\sqrt{3})}{2}\) 

 \(\boxed{x = -5 \pm i\sqrt{3}}\) 

 (No extraneous solutions)
\end{solution}\\\begin{solution}\cii\end{solution} \vspace{\stretch{1}}\\\,\answerline

\question[10] Solve: Solve for all solutions. Identify any extraneous solutions.

 \(\sqrt{x + 90} + 1 = 10\) 

\begin{solution}
\(\sqrt{x + 90} = 9\) 

 \(x + 90 = 81\) 

 \(x = 81 - 90\) 

 \(\boxed{x = -9}\) 

 (No extraneous solutions)
\end{solution}\\\begin{solution}\cii\end{solution} \vspace{\stretch{1}}\\\,\answerline

\question[10] Solve: Solve for all solutions. Identify any extraneous solutions. 

 \(\displaystyle \frac{x}{x + 1} - \frac{6}{(x + 1)(x - 5)} = \frac{-15}{x - 5}\) 

\begin{solution}
\(x(x - 5) - 6 = -15(x + 1)\) 

 \(x^{2} + 10 \, x + 9 = 0\) 

 \((x + 1)(x + 9) = 0\) 

 \(x = -9, \quad x = -1\) 

 \(x = -1 \implies \text{Extraneous}\) 

 \(\boxed{x = -9}\)
\end{solution}\\\begin{solution}\cii\end{solution} \vspace{\stretch{1}}\\\,\answerline \newpage

\question[10] Given that the function is one-to-one, find the inverse function \(f^{-1}(x)\). 

 \(f(x) = \sqrt[3]{x} - 3\) 

\begin{solution}
\(y = f(x) = \sqrt[3]{x} - 3\) 

 \(x = \sqrt[3]{y} - 3\) 

 \(x + 3 = \sqrt[3]{y}\) 

 \((x + 3)^3 = y\) 

 \(y = (x + 3)^3\) 

 \(\boxed{ f^{-1}(x) = (x + 3)^3 }\) 

  

 \(= (x + 3)(x^2 + 6x + 9)\) 

 \(\boxed{ f^{-1}(x) = x^{3} + 9 \, x^{2} + 27 \, x + 27 }\)
\end{solution}\\\begin{solution}\ci\end{solution} \vspace{\stretch{2}}\\\,\answerline \newpage

\question Suppose a rocket carrying fireworks is launched from a hill 29 feet above a lake. The rocket's height \(h\) (in feet) above the lake at time \(t\) (in seconds) is given by

 \(h(t) = -16t^2 + 160t + 29\)
\begin{parts}
  \part[2] When will the rocket reach its maximum height?

\begin{solution}
\(t = \frac{-160}{2(-16)} = \frac{-160}{-32}\) 

 \(\boxed{t = 5 \text{ sec}}\)
\end{solution}  \fillwithlines{0.5in}
  \part[2] What is the maximum height the rocket will reach?

\begin{solution}
\(h(5) = -16(5)^2 + 160(5) + 29\) 

 \(= -400 + 800 + 29\) 

 \(\boxed{h = 429 \text{ ft}}\)
\end{solution}  \fillwithlines{0.5in}
  \part[1] Interpret your results for the previous two questions in one or more complete sentences, including appropriate units.

\begin{solution}
The rocket will reach a maximum height of 429 ft 5 seconds after launch.
\end{solution}  \fillwithlines{0.5in}
\end{parts}

\question Find the following compositions:Given \(f(x) = 2 \, x\) and \(g(x) = -4 \, x^{2} + 7\), find the following:\\\begin{solution}\ci\end{solution}

\begin{parts}
  \part[6] \((f \circ g)(x)\)

\begin{solution}
\(f(g(x)) = 2(-4 \, x^{2} + 7)\) 

 \(\boxed{ -8 \, x^{2} + 14 }\) 

ERROR 1: \(g(f(x))\)

 \(= -4(2 \, x)^2 + 7\) 

 \(= -4(4 \, x^{2}) + 7\) 

 \(\boxed{ -16 \, x^{2} + 7 }\) 

ERROR 2: \(f(x) \cdot g(x)\)

 \(= (2 \, x)(-4 \, x^{2} + 7)\) 

 \(\boxed{ -8 \, x^{3} + 14 \, x }\)
\end{solution}  \vspace{\stretch{1}} \\ \answerline
  \part[4] \((f \circ f)(x)\)

\begin{solution}
\(f(f(x)) = 2(2 \, x)\) 

 \(\boxed{ 4 \, x }\)
\end{solution}  \vspace{\stretch{1}} \\ \answerline
\end{parts} \newpage

\question[5] Evaluate the difference quotient.\\\begin{solution}\ci\end{solution}
Evaluate the difference quotient, \(\displaystyle{ \frac{f(x+h)-f(x)}{h}}\) , for \(f(x) = 5 \, x^{2} + 9\). 

\begin{solution}
\(f(x+h) = 5(x+h)^2 + 9\) 

 \(= 5(x^2 + 2xh + h^2) + 9\) 

 \(= 5 \, h^{2} + 10 \, h x + 5 \, x^{2} + 9\) 

 \(\displaystyle{\frac{f(x+h)-f(x)}{h} = \frac{ (5 \, h^{2} + 10 \, h x + 5 \, x^{2} + 9) - (5 \, x^{2} + 9) }{h}}\) 

 \(=\displaystyle{ \frac{ 5 \, h^{2} + 10 \, h x }{h}}\) 

 \(= \displaystyle{ \frac{ h(5 \, h + 10 \, x) }{h}}\) 

 \(\boxed{ 5 \, h + 10 \, x }\)
\end{solution} \vspace{\stretch{1}} \\ \answerline \newpage

\question[5] Find the vertex and properties.
Find each of the properties below for the given function: 

 \(f(x) = -\frac{1}{4}x^2 +1\) 

\begin{itemize}
\item
Vertex

\item
Axis of Symmetry

\item
\(x\)-intercepts

\item
\(y\)-intercept

\item
Domain

\item
Range

\end{itemize}
\begin{solution}
\(\textbf{Direction: } \text{Opens Down}\) 

 \(\textbf{Vertex: } (0, 1)\) 

 \(\textbf{Axis of Symmetry: } x = 0\) 

 \(\textbf{x-intercepts:}\) 

 \(-\frac{1}{4}x^2 +1 = 0\) 

 \(x^2 = 4\) 

 \(x = 0 \pm 2\) 

 \(\boxed{ -2, 2 }\) 

 \(\textbf{y-intercept:}\) 

 \(= -\frac{1}{4}(0)^2 +1\) 

 \(= -\frac{1}{4}(0) +1\) 

 \(= 0 +1\) 

 \(\boxed{ 1 }\) 

 \(\textbf{Domain: } (-\infty, \infty)\) 

 \(\textbf{Range: } (-\infty, 1]\)
\end{solution}

\question[5] Use the characteristics to graph.
A quadratic function has the characteristics given below. Use the axis of symmetry to generate two additional points, then use all five points to graph the function. 

\begin{itemize}
\item
\(\text{Vertex: } (-4, -2)\)

\item
\(x\text{-intercept: } -6\)

\item
\(y\text{-intercept: } 6\)

\end{itemize}
 \begin{tikzpicture}[scale=0.35] \draw[step=1cm, gray!40, very thin] (-10,-10) grid (10,10); \draw[thick, <->] (-10.5,0) -- (10.5,0); \draw[thick, <->] (0,-10.5) -- (0,10.5); \end{tikzpicture} 

\begin{solution}
\begin{multicols}{2}  

\begin{itemize}
\item
\(\text{Vertex: } (-4, -2)\)

\item
\(y\text{-intercept: } (0, 6)\)

\item
\(\text{Reflected } y\text{-int: } (-8, 6)\)

\item
\(\text{Given Root: } (-6, 0)\)

\item
\(\text{Reflected Root: } (-2, 0)\)

\end{itemize}
 \columnbreak  

 \begin{tikzpicture}[scale=0.35] \draw[step=1cm, gray!40, very thin] (-10,-10) grid (10,10); \draw[thick, <->] (-10.5,0) -- (10.5,0); \draw[thick, <->] (0,-10.5) -- (0,10.5); \clip (-10,-10) rectangle (10,10); \draw[line width=1.5pt, blue!80!black, samples=100, domain=-10:10, <->] plot (\x, {0.5*(\x - -4)^2 + -2}); \fill[blue] (-4,-2) circle (7pt); \fill[blue] (-6,0) circle (7pt); \fill[blue] (-2,0) circle (7pt); \fill[blue] (0,6) circle (7pt); \fill[blue] (-8,6) circle (7pt); \end{tikzpicture} 

 \end{multicols}
\end{solution} \newpage

\question[10] Identify transformations and graph.
Use the table to identify the transformations described by \(g(x) = f(-(x + 5)) - 2\). 

 Circle the option that applies and fill in the blanks as appropriate to describe the transformations of the given function. If one does not apply, you may leave it blank. Then apply these transformations to the graph of the function shown below. 

 \(\renewcommand{\arraystretch}{3} \begin{array}{|l|l|} \hline \textbf{Horizontal Transformations} & \textbf{Vertical Transformations} \\ \hline \text{Reflection: YES or NO} & \text{Reflection: YES or NO} \\ \hline \text{Dilation: \underline{\hspace{2cm}} times as wide} & \text{Dilation: \underline{\hspace{2cm}} times as tall} \\ \hline \text{Translation: \underline{\hspace{2cm}} units LEFT or RIGHT} & \text{Translation: \underline{\hspace{2cm}} units UP or DOWN} \\ \hline \end{array}\) 

 \noindent\makebox[\textwidth][c]{ \begin{minipage}{0.48\textwidth} \centering \begin{tikzpicture}[scale=0.39] \draw[step=1cm, gray!40, very thin] (-10,-10) grid (10,10); \draw[thick, <->] (-10.5,0) -- (10.5,0); \draw[thick, <->] (0,-10.5) -- (0,10.5); \draw[line width=1.5pt, black] (-2,1) -- (-1,2) -- (0,0) -- (1,4) -- (3,4);\fill[black] (-2,1) circle (5pt); \fill[black] (-1,2) circle (5pt); \fill[black] (0,0) circle (5pt); \fill[black] (1,4) circle (5pt); \fill[black] (3,4) circle (5pt); \end{tikzpicture} \end{minipage} \hfill \begin{minipage}{0.48\textwidth} \centering \begin{tikzpicture}[scale=0.39] \draw[step=1cm, gray!40, very thin] (-10,-10) grid (10,10); \draw[thick, <->] (-10.5,0) -- (10.5,0); \draw[thick, <->] (0,-10.5) -- (0,10.5); \end{tikzpicture} \end{minipage} } 

\begin{solution}
\begin{itemize}
\item
 \(\textbf{Horizontal: } \text{Reflection: YES, Dilation: 1, Shift: 5 units LEFT.}\) 

\item
 \(\textbf{Vertical: } \text{Reflection: NO, Dilation: 1, Shift: 2 units DOWN.}\) 

\end{itemize}


 \begin{tikzpicture}[scale=0.39] \draw[step=1cm, gray!40, very thin] (-10,-10) grid (10,10); \draw[thick, <->] (-10.5,0) -- (10.5,0); \draw[thick, <->] (0,-10.5) -- (0,10.5); \draw[line width=1.5pt, blue] (-3,-1) -- (-4,0) -- (-5,-2) -- (-6,2) -- (-8,2);\fill[blue] (-3,-1) circle (5pt); \fill[blue] (-4,0) circle (5pt); \fill[blue] (-5,-2) circle (5pt); \fill[blue] (-6,2) circle (5pt); \fill[blue] (-8,2) circle (5pt); \end{tikzpicture}
\end{solution} \newpage


\end{questions}
\end{document}
